\documentclass[]{article}

\usepackage{graphicx}
\usepackage{float}
\usepackage[a4paper,top=2.5cm,bottom=2.5cm,left=2.5cm,right=2cm]{geometry}
\usepackage{subcaption}
\usepackage[table,xcdraw]{xcolor}
\usepackage{makecell}
\usepackage{tabularx}
\usepackage{enumitem}

\setlength{\tabcolsep}{13pt}
\renewcommand{\arraystretch}{2.3}

\title{ DD \\
	\begin{large} 
		Software Engineering 2
	\end{large}}

\author{Antonio Ercolani - 10621728\\Vittorio Fabris - 10562731\\Riccardo Nannini - 10626268}
\date{Academic year 2020/2021}


\begin{document}
	\pagenumbering{gobble}
	
	\maketitle
	

	\newpage
	\pagenumbering{arabic}	


	\tableofcontents
	
	\newpage
	
	
	\section{Introduction}
	
	\subsection{Purpose}
	
	\subsection{Scope}
	
	\begin{paragraph}
		\newline
		CLup is an application that aims to provide the users with the possibility to queue to enter in a store, preserving as much as possible their safety and health. The lining up procedure can be done in two different ways: nearby the store with a physical ticket or from the application, where a virtual ticket is generated. In this way the crowd in the neighborhood of the stores is reduced, and so as a consequence the risks to get in touch with other people is decreased too. People lining up from home will start approach the store only when the system provides them a notification. Customers will then enter the store only when their turn has come, by the recognition of the QRcode on their ticket.\\
		This application is built in order to be as easy as possible so that it can be used by customers of all the ages. Customers can simply line-up to the store they prefer, but they can also take advance of other services that are implemented inside the book a visit feature. Store managers have a different interface to deal with the system, as they have different things to check and for sure different goals, such as monitor entrances/exits and then allow more people in the store if it is possible. \\
		However, more specific and deep descriptions of the available features can be found in the RASD document.\\
		
	\end{paragraph}

	\subsection{Definitions, Acronyms, Abbreviations}
	
		In this section we explain the meaning of some technical terms used in the document.
		
		
		\subsubsection{Definitions}
		
			\medskip
			
			\begin{tabular}{|c|l|}
				\hline
				\rowcolor[HTML]{DCDCDC} 
				\textbf{QR CODE} & \makecell[l]{A \textit{Quick Response code} is a kind of bar-code, readable by machines to retrieve \\information} \\ \hline
				\textbf{prova} & \makecell[l]{prova} \\ \hline
				\rowcolor[HTML]{DCDCDC} 
				\textbf{prova2} & \makecell[l]{prova} \\ \hline
			\end{tabular}
		
		
		\subsubsection{Acronyms}
		
			\medskip
			
			\begin{tabular}{|c|l|}
				\hline
				\rowcolor[HTML]{DCDCDC} 
				\textbf{RASD} & \makecell[l]{A \textit{Quick Response code} is a kind of bar-code, readable by machines to retrieve information} \\ \hline
				\textbf{DD} & \makecell[l]{Design Document} \\ \hline
				\rowcolor[HTML]{DCDCDC} 
				\textbf{GPS}& \makecell[l]{Global Positioning System} \\ \hline
			\end{tabular}
		
		
		\subsubsection{Abbreviations}
						
			
		\subsubsection{Revision History}
		
		
		\subsubsection{Reference Documents}
		
		
		\subsubsection{Document Structure}
		
			\paragraph{Chapter 1} Describes the purposes and the scope of the DD, including the structure of the document and the set of definitions, acronyms and abbreviations used.
		
			\paragraph{Chapter 2}
		
			\paragraph{Chapter 3} 
		
			\paragraph{Chapter 4}
		
			\paragraph{Chapters 5}
		
			\paragraph{Chapters 6} Shows the effort spent for each member of the group
		
			\paragraph{Chapters 7} Includes the reference documents.
		
		
			
			
	

				
\end{document}
