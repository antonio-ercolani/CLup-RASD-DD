\documentclass[]{article}

\usepackage{graphicx}
\usepackage{float}
\usepackage[a4paper,top=2.5cm,bottom=2.5cm,left=2.5cm,right=2cm]{geometry}
\usepackage{subcaption}
\usepackage[table,xcdraw]{xcolor}
\usepackage{makecell}
\usepackage{tabularx}
\usepackage{enumitem}

\setlength{\tabcolsep}{13pt}
\renewcommand{\arraystretch}{2.3}

\title{ DD \\
	\begin{large} 
		Software Engineering 2
	\end{large}}

\author{Antonio Ercolani - 10621728\\Vittorio Fabris - 10562731\\Riccardo Nannini - 10626268}
\date{Academic year 2020/2021}


\begin{document}
	\pagenumbering{gobble}
	
	\maketitle
	

	\newpage
	\pagenumbering{arabic}	


	\tableofcontents
	
	\newpage
	
	
	\section{Introduction}
	
	\subsection{Purpose}
	The purpose of this document is defining the main design principles of the CLup software system, taking as input the concepts defined in the RASD. This document treats many topics regarding the software design. It starts from the high-level architecture choices and continues with the description of the main components, also describing how they interact with each other. The last section is about the system implementation, integration and the testing phases, useful to the developer to put together the various design aspects during the system development. 
	\\The reader can find a more detailed list of the treated topics in section 1.3.6.
	
	
	\subsection{Scope}
	
	\begin{paragraph}
		\newline
		CLup is an application that aims to provide the users with the possibility to queue to enter in a store, preserving as much as possible their safety and health. The lining up procedure can be done in two different ways: nearby the store with a physical ticket or from the application, where a virtual ticket is generated. In this way the crowd in the neighborhood of the stores is reduced, and so as a consequence the risks to get in touch with other people is decreased too. People lining up from home will start approach the store only when the system provides them a notification. Customers will then enter the store only when their turn has come, by the recognition of the QRcode on their ticket.\\
		This application is built in order to be as easy as possible so that it can be used by customers of all the ages. Customers can simply line-up to the store they prefer, but they can also take advance of other services that are implemented inside the book a visit feature. Store managers have a different interface to deal with the system, as they have different things to check and for sure different goals, such as monitor entrances/exits and then allow more people in the store if it is possible. \\
		However, more specific and deep descriptions of the available features can be found in the RASD document.\\
		
	\end{paragraph}

	\subsection{Definitions, Acronyms, Abbreviations}
	
		In this section we explain the meaning of some technical terms used in the document.
		
		
		\subsubsection{Definitions}
		
			\medskip
			
			\begin{tabular}{|c|l|}
				\hline
				\rowcolor[HTML]{DCDCDC} 
				\textbf{QR CODE} & \makecell[l]{A \textit{Quick Response code} is a kind of bar-code, readable by machines to retrieve \\information} \\ \hline
				\textbf{prova} & \makecell[l]{prova} \\ \hline
				\rowcolor[HTML]{DCDCDC} 
				\textbf{prova2} & \makecell[l]{prova} \\ \hline
			\end{tabular}
		
		
		\subsubsection{Acronyms}
		
			\medskip
			
			\begin{tabular}{|c|l|}
				\hline
				\rowcolor[HTML]{DCDCDC} 
				\textbf{RASD} & \makecell[l]{A \textit{Quick Response code} is a kind of bar-code, readable by machines to retrieve information} \\ \hline
				\textbf{DD} & \makecell[l]{Design Document} \\ \hline
				\rowcolor[HTML]{DCDCDC} 
				\textbf{GPS}& \makecell[l]{Global Positioning System} \\ \hline
			\end{tabular}
		
		
		\subsubsection{Abbreviations}
						
			
		\subsubsection{Revision History}
		
		
		\subsubsection{Reference Documents}
		
		
		\subsubsection{Document Structure}
		
			Here a list of the topics treated in each chapter of this Design Document.
			
			\paragraph{Chapter 1} is an introductory chapter, where are presented the purpose and the scope of this document. It also includes tables about acronyms and technical definitions.	
			
			\paragraph{Chapter 2} is the core of the Design Document. Here we can find the main architectural decisions, starting from the high-level design patterns. Then, there's a description of every single components and the interaction with each other. Lots of diagrams are included among the different subsections to better explain the presented concepts.
			
			\paragraph{Chapter 3} presents a deep description of the User Interface by means of a large number of detailed mockups and UX diagrams.
			
			\paragraph{Chapter 4} contains the strongest link to the RASD. In fact, it shows a mapping between the requirements presented in the RASD and the architectural components presented in chapter 2.
			
			\paragraph{Chapters 5} aims to give to the developers the guidelines of the implementation, integration and test phases, from both an high-level and a low-level point of view.
		
			\paragraph{Chapters 6 and 7} contains respectively tables about the effort spent by each group member in writing this document and the document references.
		
			
			
		\section{Effort spent}
			
			\medskip
			\textbf{\large Antonio Ercolani:} \\ \newline
			\begin{tabular}{|l|c|}
				\hline
				Purpose and Document Structure &  \textbf{1h} \\ \hline
				\rowcolor[HTML]{DCDCDC} 
				 & \textbf{} \\ \hline
				
			\end{tabular}
	
			
			\section{References}	
			
			
	

				
\end{document}
