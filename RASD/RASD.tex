\documentclass[]{article}

\usepackage{graphicx}
\usepackage[a4paper,top=2.5cm,bottom=2.5cm,left=2.5cm,right=2cm]{geometry}
\usepackage{subcaption}
\usepackage[table,xcdraw]{xcolor}
\usepackage{makecell}

\setlength{\tabcolsep}{13pt}
\renewcommand{\arraystretch}{2.3}

\title{ RASD \\
	\begin{large} 
		Software Engineering 2
	\end{large}}

\author{Antonio Ercolani - 10621728\\Vittorio Fabris - 10562731\\Riccardo Nannini - 10626268}
\date{Academic year 2020/2021}


\begin{document}
	\pagenumbering{gobble}
	
	\maketitle
	\begin{paragraph}
		\newline
	\end{paragraph}

	\newpage
	\pagenumbering{arabic}	


	\tableofcontents
	
	\newpage
	
	
	\section{Introduction}
	
	\subsection{Purpose}

	\subsection{Scope}
	
		\begin{paragraph}
			\newline
				The scope of the software is to give the users the possibility to line up to enter and shop into a grocery store.\\
				The stores that have registered to this service generate and distribute tickets to line people correctly, letting them do the shopping only if the QR-code scanned on their ticket is valid.\\
				CLup offers different kind of functionalities and features:\\
	
			\begin{itemize}
			\item \textbf{Basic:}
				The user can line up in different ways: through a physical ticket taken outside the store, or through the application, that is implemented in a way such that people of every age can use. \\
				If there are already too many people queuing outside the shop, the software doesn’t allow to distribute new physical tickets. \\
				The software can integrate the information derived  from the two sources, virtual and physical bookings, so that overcrowding is avoided in the neighborhood of the store and thanks to the QR ticket validations store managers can monitor entrances. \\
				When the user leaves the store must show his ticket to the QR-reader again, so that the system can know it and let the queue proceed, with another customer enter the shop.\\

			\item \textbf{Feature 1:}
				Users that line up from home concede the system to see where they are, using the GPS position of their device to identify their zone. In this way only nearby grocery stores are shown to the user.\\
				What’s more, these category of users can book a visit at a specific time, if there is a free spot in the schedule organized by the system, otherwise other arrival times or grocery stores are proposed to the user.\\
				The system, knowing the position of the user, his distance from the store and the vehicle that he has chosen to use to reach it, will provide him a notification that in a certain moment he should leave to arrive on time at the grocery store. If the user arrives late, the system can reschedule a booking for him, always considering to avoid overcrowding and the time preferences of the customer, inserting him into the line again.\\

			\item  \textbf{Feature 2:}
				Because the system needs to balance in the best way possible the number of people inside the store, the user that books online his visit is also asked which kind of products he’s going to buy. In this way the system can manage people along the store’s aisles, reducing the possible interaction and proximity of customers, distributing their access into the shop in different times if it is necessary.  \\

			\item  \textbf{Feature 3:}
				When the grocery store registers to the service, it can decide if he wants to save the information about the preferences of customers about the products they buy. In fact, as explained in the previous feature, the user lets the system know what he intends to buy. The grocery store could use these preferences in order to manage a plan for the refill of its shelves and always be ready to satisfy the needs of  its customers, avoiding the risk of not having some products to sell, but ordering them from their providers in advance.

			\end{itemize}
			
	
		\end{paragraph}
	
		\subsubsection{World phenomena}

			\begin{tabular}{|c|l|}
				\hline
				\rowcolor[HTML]{DCDCDC} 
				\textbf{WP1} & The customer arrives to the store \\ \hline
				\textbf{WP2} & The customer shops in the store \\ \hline
				\rowcolor[HTML]{DCDCDC} 
				\textbf{WP3} & The customer waits outside the store \\ \hline
			\end{tabular}
		\subsubsection{Shared phenomena}
			\begin{itemize}
				\item Controlled by the \textbf{machine}\newline\newline
					\begin{tabular}{|c|l|}
						\hline
						\rowcolor[HTML]{DCDCDC} 
						\textbf{SP1} & The customer is notified that his turn is coming \\ \hline
						\textbf{SP2} & The store generates and sends a ticket to the user (?) \\ \hline
					\end{tabular}
					
				\item Controlled by the \textbf{world}\newline\newline
					\begin{tabular}{|c|l|}
						\hline
						\rowcolor[HTML]{DCDCDC} 
						\textbf{SP3} & The customer enters the store scanning his QR code \\ \hline
						\textbf{SP4} & The customer reserves a place in the queue of a given store using the application \\ \hline
						\rowcolor[HTML]{DCDCDC} 
						\textbf{SP5} & \makecell[l]{The customer reserves a place in the queue of a given store using the ticket totem \\outside the store}\\ \hline
						\textbf{SP6} & The customer books a visit to the store \\ \hline
						\rowcolor[HTML]{DCDCDC} 
						\textbf{SP7} & The customer leaves the store \\ \hline
						\textbf{SP8} & The store receives a request to join its queue \\ \hline
					\end{tabular}
					
			\end{itemize}
		
		\subsubsection{Goals}
		
		\begin{tabular}{|c|l|}
						\hline
						\rowcolor[HTML]{DCDCDC} 
						\textbf{G1} & \makecell[l]{At any time the number of people in the store must not be higher than the store limit or \\ the limit is passed but the customers are expected to be in different areas of the store} \\ \hline
						\textbf{G2} & The physical queue of people outside the store is reduced \\ \hline
						\rowcolor[HTML]{DCDCDC} 
						\textbf{G3} & Allow users to "line up" for a store from home \\ \hline
						\textbf{G4} & Allow users to wait for their turn at home and receive an alert when their turn is coming \\ \hline		
						\rowcolor[HTML]{DCDCDC} 
						\textbf{G5} & Allow users to book a visit to a store \\ \hline
						\textbf{G6} & Balance the number of people between different store or timeframes with suggestions \\ \hline		
						\rowcolor[HTML]{DCDCDC} 
						\textbf{G7} & Allow the store to monitor entries \\ \hline
						\textbf{G8} & Handle the order of entries between the queue and the booked visits \\ \hline	
						\rowcolor[HTML]{DCDCDC} 
						\textbf{G9} & Collects information in order to build statistics and infer mean datas \\ \hline					
					\end{tabular}

\end{document}
